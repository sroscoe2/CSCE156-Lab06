\documentclass[12pt]{exam}
%\printanswers

\setlength{\parindent}{0pt}
\setlength{\parskip}{.25cm}

\usepackage{graphicx}

\usepackage{xcolor}

\definecolor{darkred}{rgb}{0.5,0,0}
\definecolor{darkgreen}{rgb}{0,0.5,0}
\usepackage{hyperref}
\hypersetup{
  letterpaper,
  colorlinks,
  linkcolor=red,
  citecolor=darkgreen,
  menucolor=darkred,
  urlcolor=blue,
  pdfpagemode=none,
  pdftitle={CSCE 156 Lab Handout},
  pdfauthor={Christopher M. Bourke},
  pdfsubject={},
  pdfkeywords={}
}

\definecolor{MyDarkBlue}{rgb}{0,0.08,0.45}
\definecolor{MyDarkRed}{rgb}{0.45,0.08,0}
\definecolor{MyDarkGreen}{rgb}{0.08,0.45,0.08}

\definecolor{mintedBackground}{rgb}{0.95,0.95,0.95}
\definecolor{mintedInlineBackground}{rgb}{.90,.90,1}

%\usepackage{newfloat}
\usepackage[newfloat=true]{minted}
\setminted{mathescape,
               linenos,
               autogobble,
               frame=none,
               framesep=2mm,
               framerule=0.4pt,
               %label=foo,
               xleftmargin=2em,
               xrightmargin=0em,
               startinline=true,  %PHP only, allow it to omit the PHP Tags *** with this option, variables using dollar sign in comments are treated as latex math
               numbersep=10pt, %gap between line numbers and start of line
               style=default, %syntax highlighting style, default is "default"
               			    %gallery: http://help.farbox.com/pygments.html
			    	    %list available: pygmentize -L styles
               bgcolor=mintedBackground} %prevents breaking across pages
               
\setmintedinline{bgcolor={mintedBackground}}
\setminted[text]{bgcolor={mintedBackground},linenos=false,autogobble,xleftmargin=1em}
%\setminted[php]{bgcolor=mintedBackgroundPHP} %startinline=True}
\SetupFloatingEnvironment{listing}{name=Code Sample}
\SetupFloatingEnvironment{listing}{listname=List of Code Samples}

\title{CSCE 156 -- Computer Science II \\
{\large Lab 7.0 - SQL I - Worksheet}}
%\subtitle{Lab 07 - SQL I - Worksheet}
\author{~}
\date{~}

\begin{document}

\maketitle

{\Large Names \underline{\hspace*{5cm}}}

For each question, write an SQL query to get the specified result.  You are highly 
encouraged to use a GUI SQL tool such as MySQL Workbench and keep track of your 
queries in an SQL script so that lab instructors can verify your work.  If you do, 
write your queries in the script file provided rather than hand-writing your queries 
here.

\fullwidth{\section*{Simple Queries}}

\begin{questions}

\question List all albums in the database.
\begin{solution}

\begin{minted}{sql}
select * from Album;
\end{minted}
\end{solution}

\question List all albums in the database from newest to oldest.
\begin{solution}

\begin{minted}{sql}
select * from Album order by year desc;
\end{minted}
\end{solution}

\question List all bands in the database that begin with ``The''.
\begin{solution}

\begin{minted}{sql}
select * from Band where name like 'The%';
\end{minted}
\end{solution}

\question List all songs in the database in alphabetic order.
\begin{solution}

\begin{minted}{sql}
select * from Song order by title asc;
\end{minted}
\end{solution}

\question Write a query that gives just the \mintinline{sql}{albumId} of the
album ``Nevermind''.
\begin{solution}

\begin{minted}{sql}
select albumId from Album where title = 'Nevermind';
\end{minted}
\end{solution}

\fullwidth{\section*{Simple Aggregate Queries}}

\question Write a query to determine how many musicians are in the database.
\begin{solution}

\begin{minted}{sql}
select count(*) as numMusicians from Musician;
\end{minted}
\end{solution}

\question Write a (nested) query to list the old(est) albums in the database.
\begin{solution}

\begin{minted}{sql}
select * from Album where year = (select min(year) from Album);
\end{minted}
\end{solution}

\question Write a query to find the total running time (in seconds) of all 
tracks on the album \emph{Rain Dogs} by Tom Waits

\begin{solution}

\begin{minted}{sql}
select sum(trackLength) from AlbumSong 
  where albumId = (select albumId from Album where title = 'Rain Dogs');
\end{minted}
\end{solution}

\fullwidth{\section*{Join Queries}}

\question Write a query list all albums in the database along with the album's
band, but only include the album title, year and band name.

\begin{solution}

\begin{minted}{sql}
select a.title, a.year, b.name from Album a 
  join Band b on a.bandId = b.bandId;
\end{minted}
\end{solution}

\question Write a query that lists all albums and all tracks on the albums for the
band Nirvana.

\begin{solution}

\begin{minted}{sql}
select * from Band b
  join Album a on b.bandId = a.bandId
  join AlbumSong t on t.albumId = a.albumId
  join Song s on s.songId = t.songId
  where b.name = 'Nirvana'
  order by a.title asc, t.trackNumber asc;
\end{minted}
\end{solution}

\question Write a query that list all bands along with all their albums in the
database \emph{even if they do not have any}.

\begin{solution}

\begin{minted}{sql}
select b.name, a.title, a.year from Band b
  left join Album a on a.bandId = b.bandId;
\end{minted}
\end{solution}



\fullwidth{\section*{Grouped Join Queries}}

\question Write a query list all bands along with a \emph{count} of how many
albums they have in the database (as you saw in the previous query, some should
have zero).

\begin{solution}

\begin{minted}{sql}
select b.name, count(albumId) from Band b
  left join Album a on a.bandId = b.bandId
  group by b.bandId;
\end{minted}
\end{solution}

\question Write a query that lists all albums in the database along with the
number of tracks on them.

\begin{solution}

\begin{minted}{sql}
select a.title,count(t.albumSongId) from Album a
  join AlbumSong t on t.albumId = a.albumId
  group by a.albumId;
\end{minted}
\end{solution}

\question Write the same query, but limit it to albums which have 12 or more 
tracks on them.

\begin{solution}

\begin{minted}{sql}
select a.title,count(t.albumSongId) as numTracks from Album a
  join AlbumSong t on t.albumId = a.albumId
  group by a.albumId
  having numTracks >= 12;
\end{minted}
\end{solution}


\question Write a query to find all musicians that are not in any bands.

\begin{solution}

\begin{minted}{sql}
select m.firstName,m.lastName from Musician m
  left join BandMember bm on bm.musicianId = m.musicianId
  where bm.bandMusicianId is null;
\end{minted}
\end{solution}

\question Write a query to find all musicians that are in more than one band.

\begin{solution}

\begin{minted}{sql}
select m.firstName,m.lastName,count(bandMusicianId) as numBands from Musician m
  left join BandMember bm on bm.musicianId = m.musicianId
  having numBands > 1;
\end{minted}
\end{solution}



\end{questions}


\section*{Advanced Activities}

\begin{enumerate}
  \item SQL supports basic arithmetic operations (\mintinline{sql}{+, -, /, *, %}) 
        in its queries.  Design an SQL query that calculates the total running 
        time for a particular album (identified by AlbumID) by selecting two 
        columns: minutes and seconds which both should be whole integers.  Then 
        create a query that returns the running time as a string in the format, 
        ``mm:ss'' (hint/warning: string formatting is specific to particular 
        databases and is not standard SQL; for MySQL see the \mintinline{sql}{LPAD} 
        and \mintinline{sql}{CONCAT} functions).
  \item Read up on the syntax for creating a view--a stored query that creates a 
        virtual table that can be queried as if it were an actual table in the 
        database.  Create a view in your album database to ``flatten'' the album 
        and song data into one accessible table; include the following columns: 
        \mintinline{sql}{albumId, albumTitle, bandId, trackNumber, songId, songTitle}.
\end{enumerate}


\end{document}
